\chapter{Estado de la cuestión}  
\addcontentsline{toc}{chapter}{\numberline{}Estado de la cuestión}

\section{Marco teórico del trabajo}

\subsection{Esquemas de cifrado simétrico}

\begin{definition}
	Un \textbf{esquema de cifrado simétrico} está dado por
	\begin{enumerate}
		\item Un conjunto $P$ denominado \textbf{espacio de textos planos} y otro conjunto $C$ denominado \textbf{espacio de textos cifrados}.
		\item Un conjunto $K$ denominado \textbf{espacio de claves}.
		\item Un conjunto $\{E_e : P \rightarrow C \mid e \in K\}$ cuyos elementos se denominan \textbf{transformaciones de cifrado} y un conjunto $\{D_d : C \rightarrow P \mid d \in K\}$ cuyos elementos se denominan \textbf{transformaciones de descifrado} tales que existe una clave $\kappa \in K^2$ con la cual para todo $p \in P$ se cumple $D_d(E_e(p)) = p$.
	\end{enumerate}
\end{definition}
%\footnote{Si bien este concepto puede ser definido de una manera más rigurosa, ya que esto queda fuera del ámbito de este trabajo, entiéndase por \textit{computacionalmente difícil} que el número de operaciones que hacen falta para obtener $d$ a partir de $e$ es tan grande que resulta inviable llevarlo a la práctica, y por \textit{computacionalmente fácil} que no sea computacionalmente difícil \autocite{Pellikaan_Wu_Bulygin_Jurrius_2017}.}
\subsection{Esquemas de cifrado asimétrico}

\subsubsection{Fundamentos}

\begin{definition}
	Una función $f : X \rightarrow Y$ es \textbf{unidireccional} si para un $x \in X$ es computacionalmente fácil calcular $f(x)$ pero para un $y \in Im(f)$ es computacionalmente difícil obtener un valor $x$ tal que $f(x) = y$.
\end{definition}

\begin{example} \textbf{(Problema del logaritmo discreto (DLP)).}
	Sea $g : \mathbb{F}_q^* \rightarrow \mathbb{F}_q^*$ tal que $g(x) = a^x$ y $a$ genere el grupo multiplicativo $\mathbb{F}_q^*$. Dado $x$, es computacionalmente fácil calcular $g(x)$. Sin embargo, se cree que obtener una función inversa de $g$ es un problema computacionalmente difícil. Si este es el caso, $g$ se trata de una función unidireccional.
\end{example}

\begin{definition}
	Una función unidireccional $f : X \rightarrow Y$ es \textbf{unidireccional con trampa} si existe información adicional con la cual sea computacionalmente fácil para un $y \in Im(f)$ obtener un valor $x$ tal que $f(x) = y$.
\end{definition}

\begin{definition}
	Un \textbf{esquema de cifrado asimétrico} está dado por
	\begin{enumerate}
		\item Un conjunto $P$ denominado \textbf{espacio de textos planos} y otro conjunto $C$ denominado \textbf{espacio de textos cifrados}.
		\item Un conjunto $K$ denominado \textbf{espacio de claves}.
		\item Un conjunto $\{E_e : P \rightarrow C \mid e \in K\}$ cuyos elementos se denominan \textbf{transformaciones de cifrado} y un conjunto $\{D_d : C \rightarrow P \mid d \in K\}$ cuyos elementos se denominan \textbf{transformaciones de descifrado} tales que, dados un par $(e, d) \in K^2$ con los cuales para todo $p \in P$ se cumple $D_d(E_e(p)) = p$, es computacionalmente difícil obtener $d$ a partir de $e$.
	\end{enumerate}
\end{definition}

\subsubsection{Esquema RSA}

Uno de los esquemas de cifrado asimétrico más frecuentemente utilizados en la actualidad se conoce como RSA. Consiste en tres algoritmos: uno para generar las claves, otro para el cifrado de mensajes, y otro para el descifrado.

\begin{algorithm}
	\caption{Generación de claves}\label{alg:1}
	\KwData{Dos primos $p$ y $q$ cualesquiera.}
	\KwResult{El par de clave pública y clave privada $((e, n), d)$.}
	$n \gets pq$.
	$\phi(n) = (p - 1)(q - 1)$.
	Seleccionar un valor $e$ tal que $1 < e < \phi(n)$ y el menor común múltiplo entre $e$ y $\phi(n)$ sea $1$.
	Utilizar el algoritmo euclideano extendido para calcular un valor $d$ tal que $ed \equiv 1 \mod \phi(n)$.
\end{algorithm}

\begin{algorithm}
	\caption{Cifrado}\label{alg:2}
	\KwData{Un mensaje en claro $p$ y los valores $n$ y $e$ obtenidos en el Algoritmo\,1.}
	\KwResult{El mensaje cifrado $c$.}
	Computar $m \in [0, p) \cap \integers$ mediante una biyección $f$ tal que $m = f(p)$ de forma que $m < n$.
	Calcular $c = m^e \mod n$.
\end{algorithm}

\begin{algorithm}
	\caption{Descifrado}\label{alg:3}
	\KwData{Un mensaje cifrado $c$ y los valores $n$ y $d$ obtenidos en el Algoritmo\,1.}
	\KwResult{El mensaje en claro $p$.}
	Calcular $m = c^d \mod n$.
	Computar $p = f^{-1}(m)$, siendo $f$ la misma biyección utilizada en el Algoritmo 2.
\end{algorithm}

\newpage

Debido a que el texto cifrado es un valor módulo $n$, la longitud del contenido a cifrar es una limitación de RSA. Este es uno de los motivos por los cuales, en la práctica, el algoritmo no se utiliza para cifrar el mensaje en sí, sino para cifrar las claves del esquema de cifrado simétrico que realmente se utilizan para cifrar el mensaje.

\begin{example}
	Ejemplo con valores numéricos.
\end{example}

\subsubsection{Vulnerabilidades}

\begin{definition}
	Sea $f : X \rightarrow Y$ para el que existe un valor $r$ tal que $f(x + r) = f(x)$ para todo $x \in X$. Entonces, $f$ es una \textbf{función periódica} y $r$ es el \textbf{periodo} de $f$.
\end{definition}

Shor propuso un algoritmo cuántico para calcular el periodo de una función periódica de forma eficiente \autocite{Shor_1997}. Asimismo, Miller mostró que el problema de factorización se reduce a calcular el periodo de una función \autocite{MILLER1976300}. Esto significa que, en un ordenador cuántico hipotético, el problema de factorización podría dejar de considerarse \textit{computacionalmente difícil} como se creía.

\begin{algorithm}
	\caption{Factorización por método de Shor}\label{}
	\KwData{El número $N$ a factorizar.}
	\KwResult{Los factores $p$ y $q$.}
	Escoger un valor $a$ al azar tal que $1 < a < N$\;
	Calcular $d = \mathrm{mcd}(a, N)$ con el algoritmo de Euclides\;
	\eIf{$d \neq 1$}
	{
		Asignar $p = d$ y $q = N/p$\;
	}
	{
		Encontrar el periodo $r$ de $a$, de forma que $a^r \equiv 1 \mod N$\;
		\eIf{$r$ es impar}
		{
			Volver a empezar el algoritmo escogiendo un valor $a$ distinto\;
		}
		{
			Calcular $d' = \mathrm{mcd}(a^{\frac{r}{2}} + 1, N)$\;
			\eIf{$d' \neq 1$}
			{
				Asignar $p = a^{\frac{r}{2}} + 1$ y $q = a^{\frac{r}{2}} - 1$\;
			}
			{
				Volver a empezar el algoritmo escogiendo un valor $a$ distinto\;
			}
		}
	}
\end{algorithm}

\newpage

\subsection{Fundamentos de la teoría de códigos}

\subsubsection{Códigos de bloque}

\begin{definition}
	Un \textbf{alfabeto} es un conjunto finito no vacío cuyos elementos son denominados \textbf{letras} o \textbf{símbolos}.
\end{definition}

\begin{definition}
	Una \textbf{palabra} sobre un alfabeto $A$ es una $n$-tupla formada por símbolos de $A$. A $n$ se la denomina \textbf{longitud} de la palabra. Si $\textbf{x}$ es una palabra, se denota por $\textbf{x}_i$ la letra de $\textbf{x}$ en la posición $i$.
\end{definition}

\begin{definition}
	Un \textbf{código} es un conjunto de palabras sobre un mismo alfabeto. Un \textbf{código de bloque} es un código cuyas palabras tienen todas la misma longitud.
\end{definition}

\begin{example}
	Sea $A = \mathbb{F}_2$ el alfabeto binario. El conjunto de todas las palabras binarias de longitud $5$ es el código de bloque $C = \mathbb{F}_2^5$. La palabra $\textbf{x} = (0, 1, 1, 0, 1)$ cumple $\textbf{x} \in C$.
\end{example}

\begin{example} \textbf{(Código ISBN).}
	El código ISBN es un identificador único para libros, donde los nueve primeros dígitos de una palabra tienen valor entre $0$ y $9$ y contienen información sobre el código del país, la lengua de origen, el editor o el número del artículo; y el décimo es un \textbf{dígito de control} que verifica $\sum_{i=1}^{10} ia_i \equiv 0 \mod 11$. En términos matemáticos, $C_{\textrm{ISBN}}\subset\mathbb{F}_{11}^{10}$ es un código de bloque de longitud $10$ tal que
	\[C_{\textrm{ISBN}} = \{\textbf{x}\in\mathbb{F}_{11}^{10} \mid (a_i = 10 \implies i = 10) \wedge \sum_{i=1}^{10}ia_i \equiv 0 \mod 11\}.\]
	\begin{remark}
		Cuando $a_{10} = 10$, suele representarse con el símbolo $\textrm{X}$, pero a efectos de cálculo toma valor $10$. Por ejemplo, $\textrm{417339361X} \in C_{\textrm{ISBN}}$.
	\end{remark}
\end{example}

\begin{definition}
	Sean $\textbf{x}$ e $\textbf{y}$ dos palabras de misma longitud. La \textbf{distancia de Hamming} entre ambas es igual al número de letras en que difieren. Dicho de otra forma, si $\textbf{x} = (x_1, \hdots, x_n)$ y $\textbf{y} = (y_1, \hdots, y_n)$, entonces
	\[d(\textbf{x}, \textbf{y}) = |\{i\in[1, n]\cap\naturals \mid x_i \neq y_i\}|,\]
	donde $d(\textbf{x}, \textbf{y})$ es la distancia de Hamming entre $\textbf{x}$ e $\textbf{y}$.
\end{definition}

\begin{theorem}
	La distancia de Hamming entre  $\textbf{x}$ e $\textbf{y}$ tales que $\textbf{x}, \textbf{y} \in A^n$ es una distancia definida en $A^n$.
\end{theorem}

\begin{proof}
	Se demuestran a continuación las cuatro propiedades de las distancias:
	\begin{enumerate}
		\item \textit{La distancia de una palabra a sí misma es $0$}: en efecto,
		\[d(\textbf{x}, \textbf{x}) = |\{i\in[1,n]\cap\naturals \mid x_i \neq x_i\}| = |\{\}| = 0.\]
		\item \textit{La distancia entre dos palabras distintas es positiva}: si $\textbf{x}$ e $\textbf{y}$ son palabras distintas, significa que las dos tuplas deben diferir en al menos una de las posiciones; es decir, debe existir al menos un valor de $i$ tal que $x_i \neq y_i$, por lo que la cardinalidad de $\{i \in [1, n]\cap\naturals \mid x_i \neq y_i\}$ será como mínimo de $1$.
		\item \textit{La distancia de $\textbf{x}$ a $\textbf{y}$ es la misma que de $\textbf{y}$ a $\textbf{x}$}: se verifica
		\[d(\textbf{x}, \textbf{y}) = |\{i\in[1,n]\cap\naturals \mid x_i \neq y_i\}| = |\{i\in[1,n]\cap\naturals \mid y_i \neq x_i\}| = d(\textbf{y}, \textbf{x}).\]
		\item \textit{Se cumple la desigualdad triangular $d(\textbf{x}, \textbf{z}) \leq d(\textbf{x}, \textbf{y}) + d(\textbf{y}, \textbf{z})$}:
		Se define
		\[\Delta(\textbf{x}, \textbf{y}) = \{i\in[1,n]\cap\naturals \mid x_i \neq y_i\}.\]
		
		Supóngase que existen $\textbf{x}, \textbf{y}, \textbf{z} \in A^n$ tales que $d(\textbf{x}, \textbf{z}) > d(\textbf{x}, \textbf{y}) + d(\textbf{y}, \textbf{z})$. Entonces, debe existir $i \in [1, n]\cap\naturals$ tal que $i \in \Delta(\textbf{x}, \textbf{z})$, $i \notin \Delta(\textbf{x}, \textbf{y})$ e $i \notin \Delta(\textbf{y}, \textbf{z})$. Eso implica $x_i \neq z_i$ a la vez que $x_i = y_i = z_i$, lo cual es una contradicción.
	\end{enumerate}
\end{proof}

\begin{definition}
	Sea $\textbf{x}$ una palabra de longitud $n$. Se llama \textbf{peso de $\textbf{x}$}, y se denomina por $w(\textbf{x})$, al número de elementos de $\textbf{x}$ distintos de $0$.
\end{definition}

\begin{definition}
	El \textbf{peso mínimo} de un código $C$, denominado por $w(C)$, es el menor de los pesos no nulos entre las palabras diferentes de $C$. Es decir,
	\[w(C) = \min\{w(\textbf{x}) \mid \textbf{x}\in (C - \mathbf{0})\}.\]
\end{definition}

\begin{definition}
	La \textbf{distancia mínima} de un código $C \subseteq A^n$ es la menor de las distancias entre palabras diferentes de $C$ si $|C| > 1$, y $n+1$ en caso contrario; es decir, para códigos formados por más de una palabra,
	\[d(C) = \min\{d(\textbf{x}, \textbf{y}) \mid \textbf{x}, \textbf{y} \in C, \textbf{x}\neq\textbf{y}\}.\]
	
	\begin{remark}
		Se denotará la distancia mínima simplemente por $d$ cuando por contexto esté claro a qué código corresponde. Así, por norma general, $d = d(C)$.
	\end{remark}
\end{definition}

\begin{definition}
	Sea $\textbf{x} \in A^n$. La \textbf{bola cerrada de centro $\textbf{x}$ y radio $r$} se denota por $B_r(\textbf{x})$ y se define como
	\[B_r(\textbf{x}) = \{\textbf{y} \in A^n \mid d(\textbf{x}, \textbf{y}) \leq r\}.\]
	Similarmente, se denota por $S_r(\textbf{x})$ a la \textbf{esfera de centro $\textbf{x}$ y radio $r$}, y queda definida como
	\[S_r(\textbf{x}) = \{\textbf{y} \in A^n \mid d(\textbf{x}, \textbf{y}) = r\}.\]
	\begin{remark}
		Ya que la distancia de Hamming devuelve únicamente números naturales, es inmediata la siguiente afirmación:
		\[B_r(\textbf{x}) = \bigcup_{i=0}^r S_i(\textbf{x}).\]
	\end{remark}
\end{definition}

\begin{theorem}
	Sea $A$ un alfabeto con $q$ elementos y $\textbf{x} \in A^n$. Entonces,
	\[|B_r(\textbf{x})| = \sum_{i=0}^r\binom{n}{i}(q - 1)^i.\]
\end{theorem}

\begin{proof}
	Se define
	\[\Delta(\textbf{x}, \textbf{y}) = \{i\in[1,n]\cap\naturals \mid x_i \neq y_i\}.\]
	Sea $\textbf{y} \in S_i(\textbf{x})$. Por definición, $i = |\Delta(\textbf{x}, \textbf{y})|$.
	
	Ya que las palabras tienen longitud $n$, y se realizan $i$ cambios para llegar de $\textbf{x}$ a $\textbf{y}$, existen exactamente $\binom{n}{i}$ conjuntos $\Delta(\textbf{x}, \textbf{y})$ que verifican las ecuaciones de arriba.
	
	Además, para un $\Delta(\textbf{x}, \textbf{y})$ arbitrario, existen $(q - 1)^i$ posibles palabras $\textbf{y} \in S_i(\textbf{x})$ que verifican las ecuaciones.
	
	Así, el número de $\textbf{y} \in S_i(\textbf{x})$ que podrían verificar las ecuaciones es $\binom{n}{i}(q - 1)^i$; es decir,
	\[|S_i(\textbf{x})| = \binom{n}{i}(q - 1)^i.\]
	
	Finalmente, por la \textit{Nota 3}, se concluye que
	\[|B_r(\textbf{x})| = \bigg|\bigcup_{i=0}^r S_i(\textbf{x})\bigg| = \sum_{i=0}^r |S_i(\textbf{x})| = \sum_{i=0}^r\binom{n}{i}(q - 1)^i.\]
\end{proof}

\subsubsection{Códigos lineales}

\begin{definition}
	Un \textbf{código lineal} es un código de bloque con estructura de subespacio vectorial; es decir, para un código lineal $C$ sobre $\mathbb{F}_q^n$ se cumplen las siguientes propiedades:
	\begin{enumerate}
		\item El código $C$ no es vacío.
		\item Para todos $\textbf{x}, \textbf{y} \in C$ se verifica $\textbf{x} + \textbf{y} \in C$.
		\item Para todos $\lambda \in \mathbb{F}_q, \textbf{x} \in C$ se verifica $\lambda\textbf{x} \in C$.
	\end{enumerate}

	A un código lineal de longitud $n$ y dimensión $k$ se lo denomina \textbf{$(n, k)$-código}.
\end{definition}

\begin{theorem}
	Sea $C$ un código lineal. Se verifica $d(C) = w(C)$.
\end{theorem}

\begin{proof}
	Sean $\textbf{x}, \textbf{y} \in C$. Ya que $C$ es un código lineal, se cumple $\textbf{x} - \textbf{y} \in C$. Obsérvese que $\textbf{x}_i = \textbf{y}_i$ implica $x_i - y_i = 0$, luego $w(\textbf{x} - \textbf{y}) = d(\textbf{x}, \textbf{y})$. Así, la distancia entre cualesquiera dos palabras es siempre igual al peso de otra, y el peso de cualquier palabra es siempre igual a la distancia entre dos otras palabras, por lo que el valor menor debe ser igual en ambos casos.
\end{proof}

\begin{definition}
	Una \textbf{base} de un $(n, k)$-código $C$ es una $k$-tupla de palabras de $C$ tal que cada palabra de $C$ es combinación lineal de los elementos de la base.
\end{definition}

\begin{definition}
	Una \textbf{matriz generadora} de un $(n, k)$-código es una matriz cuyas filas forman una base de $C$.
\end{definition}

\subsubsection{Códigos cíclicos}

\begin{definition}
	El \textbf{desplazamiento cíclico} de una palabra $\textbf{x} \in \mathbb{F}_q^n$, denotado por $\sigma(\textbf{x})$, es una transformación tal que
	\[\sigma(\textbf{x}) = (\textbf{x}_{n-1}, \textbf{x}_0, \textbf{x}_1, \dots, \textbf{x}_{n-2}).\]
\end{definition}

\begin{definition}
	Un \textbf{código cíclico} es un código que contiene el desplazamiento cíclico de cada una de sus palabras.
\end{definition}

\begin{remark}
	Los códigos cíclicos pueden entenderse como un ideal. Sea $\textbf{g} \in C$. Se denota por $\textbf{g}(X)$ al polinomio
	\[\textbf{g}(X) = \textbf{g}_1X^{n-1} + \hdots \textbf{g}_{n-1}X + \textbf{g}.\]
	
	Sea $\mathbb{F}_q[X] / (X^n - 1)$ el anillo cociente del anillo de polinomios $\mathbb{F}_q[X]$ módulo $X^n - 1$. En este anillo, se verifica
	\[X^n \equiv 1 \mod X^n - 1.\]
	
	Así, se observa que aplicar un desplazamiento cíclico a un código $C$ es equivalente a multiplicar los polinomios asociados a cada palabra por $X$.
\end{remark}

%Aunque la siguiente definición no es parte propia de la teoría de códigos, sí será útil conocerla para aplicarla después a los códigos cíclicos.

%\begin{definition}
%El \textbf{polinomio mínimo} de un elemento $x$ de una extensión de $F$ es el polinomio mónico $p(X) \in F[X]$ de menor grado tal que $p(x) = 0$.
%\end{definition}

\begin{definition}
	Un \textbf{código polinómico} es un código lineal cuyas palabras son polinomios divisibles por un cierto polinomio dado. A dicho polinomio se lo denomina \textbf{polinomio generador} del código.
\end{definition}

\begin{definition}
	El \textbf{polinomio generador} de un código cíclico es el polinomio mónico de menor grado entre los polinomios asociados a palabras del código \autocite{VERLINDE2003203}.
\end{definition}

\subsubsection{Códigos polinómicos}

\begin{definition}
	Sea $\alpha$ un elemento primitivo de $\mathbb{F}_q$; es decir, que cada elemento no nulo de $\mathbb{F}_q$ puede representarse como $\alpha^i \mod q$ con $i \in \naturals$. Sea $n = q - 1$. Sean $b, k \in \naturals$ tales que $0 \leq b$ y $k \leq n$. Sea $g_{b,k}(X)$ un polinomio generador tal que
	\[g_{b,k}(X) = (X - \alpha^b)\hdots(X - \alpha^{b+n-k-1}).\]
	Se denomina \textbf{código de Reed-Solomon}, y se denota por $RS_k(n, b)$, al código cíclico $q$-ario con polinomio generador $g_{b,k}(X)$.
\end{definition}

\subsubsection{Códigos de Goppa}

\begin{definition}
	Sea $L = (a_1, \dots, a_n)$ una tupla de tal que $L \in \mathbb{F}_{q^m}^n$. Un \textbf{polinomio de Goppa} con respecto a $L$ es un polinomio $g(X) \in \mathbb{F}_{q^m}[X]$ tal que $g(a_i) \neq 0$ para cualquier $i \in [1, n] \cap \naturals$.
\end{definition}

\begin{definition}
	Un \textbf{código de Goppa} con respecto a $L$ y $g$, denotado por $\Gamma(L, g)$, es un código lineal tal que
	\[\Gamma(L, g) = \bigg\{\textbf{x} \in \mathbb{F}_q^n \ \ \bigg|\ \ \sum_{j=1}^n \frac{\textbf{x}_j}{X - a_j} \equiv 0 \mod g(X)\bigg\}.\]
\end{definition}

\subsection{Algoritmos de cifrado basados en la teoría de códigos}

\subsubsection{HQC.PKE}

\autocite{HQC}

\begin{algorithm}
	\caption{Generación de claves}\label{alg:1}
	\KwData{La longitud del código $n$, la dimensión $k$, la capacidad de corrección de errores $t$ y un peso $w$.}
	\KwResult{El par de clave pública y clave privada $(P_K, S_K)$.}
	Seleccionar aleatoriamente un polinomio $h(X) \in \mathbb{F}_q[X] / (X^n - 1)$\;
	Seleccionar una matriz $G \in \mathbb{F}_q^{k \times n}$\;
	Seleccionar aleatoriamente dos polinomios $x(X), y(X) \in \mathbb{F}_q[X] / (X^n - 1)$ tales que su peso es igual a $w$\;
	Asignar $s(X) = x(X) + h(X) \cdot y(X)$\;
	Asignar $S_K = (\textbf{x}, \textbf{y})$\;
	Asignar $P_K = (\textbf{h}, \textbf{s})$\;
\end{algorithm}

\begin{algorithm}
	\caption{Cifrado}\label{alg:1}
	\KwData{Un mensaje en claro $\textbf{p}$, la clave pública $P_K$, dos pesos $w_r$ y $w_e$ y la longitud del mensaje $l$.}
	\KwResult{El mensaje cifrado $\textbf{c}$.}
	Computar $m \in [0, p) \cap \integers$ mediante una biyección $f$ tal que $m = f(p)$\;
	Seleccionar aleatoriamente $e(X) \in \mathbb{F}_q[X] / (X^n - 1)$ tal que su peso sea igual a $w_e$\;
	Seleccionar aleatoriamente $r_1(X), r_2(X) \in \mathbb{F}_q[X] / (X^n - 1)$ tales que su peso es igual a $w_r$\;
	Asignar $u(X) = r_1(X) + h(X) \cdot r_2(X)$\;
	Asignar $v = \mathtt{truncar}(mG + s \cdot r_2 + e, l)$\;
	Asignar $\textbf{c} = (u, v)$\;
\end{algorithm}

\begin{algorithm}
	\caption{Descifrado}\label{alg:1}
	\KwData{Un mensaje cifrado $\textbf{c}$ y la clave privada $S_K$.}
	\KwResult{El mensaje en claro $\textbf{p}$.}
	Calcular $m$ decodificando $v - u \cdot y$ con un algoritmo que corrija al menos $t$ errores.
	Computar $\textbf{p} = f^{-1}(\textbf{m})$\; 
\end{algorithm}

\subsubsection{Classic McEliece}

\autocite{McEliece}

\begin{algorithm}
	\caption{Generación de claves}\label{alg:1}
	\KwData{La longitud del código $n$, la dimensión $k$, la cardinalidad del alfabeto $q$, la capacidad de corrección de errores $t$ y una clase $\mathcal{C}$ de $(n, k)$-códigos en $\mathbb{F}_q^n$ que posea un decodificador que pueda corregir hasta $t$ errores de manera eficiente.}
	\KwResult{El par de clave pública y clave privada $(P_K, S_K)$.}
	Seleccionar $C \in \mathcal{C}$. Denomínese $G$ a su matriz generadora y $D_C$ a su decodificador\;
	Seleccionar una matriz $S$ con elementos en $\mathbb{F}_q$ de dimensiones $k \times k$\;
	Seleccionar una matriz de permutaciones $P$ de dimensiones $n \times n$\;
	Calcular $G' = SGP$\;
	Asignar $P_K = G'$\;
	Asignar $S_K = (D_C, S, P)$\;
\end{algorithm}

\begin{algorithm}
	\caption{Cifrado}\label{alg:1}
	\KwData{Un mensaje en claro $\textbf{p}$ y la clave pública $P_K$.}
	\KwResult{El mensaje cifrado $\textbf{c}$.}
	Computar $m \in [0, p) \cap \integers$ mediante una biyección $f$ tal que $m = f(p)$\;
	Seleccionar de forma aleatoria $\textbf{e} \in \mathbb{F}_q^n$ de peso $t$\;
	Calcular $\textbf{c} = \textbf{m}G' + \textbf{e}$\;
\end{algorithm}

\begin{algorithm}
	\caption{Descifrado}\label{alg:1}
	\KwData{Un mensaje cifrado $\textbf{c}$ y la clave privada $S_K$.}
	\KwResult{El mensaje en claro $\textbf{p}$.}
	Calcular $\textbf{m} = D_C(\textbf{c}P^{-1})S^{-1}$\;
	Computar $\textbf{p} = f^{-1}(\textbf{m})$\; 
\end{algorithm}

\subsubsection{BIKE}

\section{Trabajos relacionados}

Existen varios trabajos cuyos 

\subsection{Criptosistema de McEliece}
El TFG «Criptosistema de McEliece» \autocite{McEliece2021} tiene como objetivo indagar a fondo en el algoritmo de McEliece clásico.

En el comienzo del trabajo, García explica el estado del arte de la computación cuántica, así como sus consecuencias en cuanto a la seguridad criptográfica. También investiga varios algoritmos presentados en el concurso del NIST.

Después, el trabajo se centra concretamente en el criptosistema de McEliece. Para ello, ofrece una breve introducción a la teoría de códigos, incidiendo en los códigos Goppa, y cómo con ella se puede definir el criptosistema. Finalmente, muestra una implementación del algoritmo y de varios ataques habituales al mismo.

\subsection{Criptografía basada en códigos}

En \autocite{Codes2024}, al igual que en el anterior trabajo de fin de grado, se comienza compartiendo algunas nociones sobre la teoría de códigos y sobre los problemas $NP$-completos, y se indaga en el criptosistema de McEliece.

El contenido de mayor interés en este documento son las nociones sobre seguridad que se ofrecen, así como tablas con niveles de seguridad estimados para distintos valores de los parámetros del criptosistema.

\subsection{McEliece-type cryptosystems: costs, security and an attack to a recent variant}

Esta tesis \autocite{costs} busca analizar numerosas variaciones de los criptosistemas de McEliece, Niederreiter y Sidel'nikov. Comienza con un repaso de la teoría de códigos para continuar con su aplicación en los criptosistemas. En el estudio, se muestra cuáles variaciones son seguras y cuáles han podido ser atacadas de manera eficiente, o podrían serlo utilizando ordenadores cuánticos.

Después, se discute el coste temporal y espacial de las distintas variaciones, y su seguridad ante diversos ataques.

\subsection{Analysis and Implementation of the Post-Quantum Cryptographic Scheme: Classic McEliece}

Esta tesis \autocite{implementation} tiene como objetivo analizar el criptosistema de McEliece en cuanto a implementación y resistencia a ataques, proponer algunas modificaciones al algoritmo, y ofrecer recomendaciones sobre los valores de parámetros que deberían usarse.

Se comienza con unos preliminares de teoría de códigos, sigue con el análisis e implementación del criptosistema original y modificado, y después se ofrece un estudio sobre la seguridad frente a un tipo concreto de ataque, los ataques de temporización.

\subsection{Status Report on the Fourth Round of the NIST Post-Quantum Cryptography Standardization Process}

Este informe del NIST \autocite{NIST8545} hace un repaso sobre todos los algoritmos que participaron en la cuarta ronda del programa de estandarización organizado por el mismo instituto: HQC, BIKE, McEliece Clásico y SIKE.

Además de explicar su funcionamiento, se comparan los algoritmos entre sí según su nivel de seguridad y su coste computacional.

Cabe destacar que, aunque HQC fuera el único de los cuatro en ser seleccionado, uno de los principales motivos por los que el algoritmo de McEliece Clásico no lo fue es que ya está en proceso de estandarización por la Organización Internacional de Normalización \autocite{isoMcEliece}, y de hecho el documento no descarta considerar una estandarización compatible en un futuro.

\subsection{Side-Channel Attacks on the McEliece and Niederreiter Public-Key Cryptosystems}

En \autocite{Avanzi2011}, se analiza concretamente la vulnerabilidad del criptosistema de McEliece frente a ataques de canal lateral. Se explica cómo el atacante podría aprovechar fugas de información en el algoritmo original, y qué cambios podrían hacerse a la implementación para aumentar la robustez del sistema frente a los ataques.

Además, incluye algunas pruebas de rendimiento de la nueva implementación con dos conjuntos de parámetros.

\subsection{Comparación de los estudios}

La mayoría de los trabajos relacionados se centran en analizar el criptosistema de McEliece. Algunos, como \autocite{McEliece2021, implementation}, van más allá de explicar la estructura del algoritmo, y se adentran en analizar también ataques que pueden utilizarse contra el sistema.

El más relevante para el presente trabajo es el de Avanzi et. al. \autocite{Avanzi2011}, que excusivamente estudia la vulnerabilidad del criptosistema frente a ataques de canal lateral, y propone modificaciones a la implementación para combatir el ataque. Aunque se incluyen pruebas de rendimiento para la implementación propuesta, estas se realizaron con escasos conjuntos de parámetros, y no llegan a explorar todos los propuestos por el equipo de McEliece clásico para el programa del NIST \autocite{NIST8545}.

Así, este Trabajo de Fin de Grado tiene como objetivo realizar una comparativa del rendimiento al ejecutar una implementación tradicional y al ejecutar la implementación con protecciones frente a este tipo de ataques, con todos los parámetros que hasta la fecha han sido más comúnmente recomendados.
