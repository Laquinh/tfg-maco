\chapter{Estado de la cuestión}  
\addcontentsline{toc}{chapter}{\numberline{}Estado de la cuestión}

\section{Marco teórico del trabajo}

\subsection{Criptografía actual}

\subsubsection{Esquemas de cifrado asimétrico}

Explicar la matemática que hay detrás (DLP), explicar RSA, ElGamal.

\subsubsection{Vulnerabilidades}

Explicar ¿cómo? ¿por qué? funciona el algoritmo de Shor.

\subsection{Fundamentos de la teoría de códigos}

\subsubsection{Códigos de bloque}

\subsubsection{Códigos lineales}

\subsubsection{Códigos cíclicos}

\subsubsection{Códigos polinómicos}

\subsubsection{Códigos Goppa}

\section{Trabajos relacionados}

(Sobre los algoritmos BIKE, Classic McEliece y HQC).
