\chapter{Estado de la cuestión}  
\addcontentsline{toc}{chapter}{\numberline{}Estado de la cuestión}

\section{Marco teórico del trabajo}

\subsection{Esquemas de cifrado simétrico}

\begin{definition}
	Un \textbf{esquema de cifrado simétrico} está dado por
	\begin{enumerate}
		\item Un código $P$ denominado \textbf{espacio de textos planos} y otro código $C$ denominado \textbf{espacio de textos cifrados}.
		\item Un conjunto $K$ denominado \textbf{espacio de claves}.
		\item Un conjunto $\{E_e : P \rightarrow C \mid e \in K\}$ cuyos elementos se denominan \textbf{transformaciones de cifrado} y un conjunto $\{D_d : C \rightarrow P \mid d \in K\}$ cuyos elementos se denominan \textbf{transformaciones de descifrado} tales que, dados un par $(e, d) \in K^2$ con los cuales para todo $p \in P$ se cumple $D_d(E_e(p)) = p$, es computacionalmente fácil\footnote{Si bien este concepto puede ser definido de una manera más rigurosa, ya que esto queda fuera del ámbito de este trabajo, entiéndase por \textit{computacionalmente difícil} que el número de operaciones que hacen falta para obtener $d$ a partir de $e$ es tan grande que resulta inviable llevarlo a la práctica, y por \textit{computacionalmente fácil} que no sea computacionalmente difícil.} obtener $d$ a partir de $e$ y vice versa.
	\end{enumerate}
\end{definition}

\subsection{Esquemas de cifrado asimétrico}

\subsubsection{Fundamentos}

\begin{definition}
	Una función $f : X \rightarrow Y$ es \textbf{irreversible} si para un $x \in X$ es computacionalmente fácil calcular $f(x)$ pero para un $y \in Im(f)$ es computacionalmente difícil obtener un valor $x$ tal que $f(x) = y$, y por \textit{computacionalmente fácil} que este caso no ocurra.
\end{definition}

\begin{definition}
	Una función irreversible $f : X \rightarrow Y$ es \textbf{irreversible con trampa} si existe información adicional con la cual sea computacionalmente fácil para un $y \in Im(f)$ obtener un valor $x$ tal que $f(x) = y$.
\end{definition}

\begin{example} \textbf{(Problema del logaritmo discreto (DLP)).}
Sea $g : \mathbb{F}_q^* \rightarrow \mathbb{F}_q^*$ tal que $g(x) = a^x$ y $a$ genere el grupo multiplicativo $\mathbb{F}_q^*$. Dado $x$, es computacionalmente fácil calcular $g(x)$. Sin embargo, se cree que obtener una función inversa de $g$ es un problema computacionalmente difícil. Si este es el caso, $g$ se trata de una función irreversible.
\end{example}

\begin{definition}
	Un \textbf{esquema de cifrado asimétrico} está dado por
	\begin{enumerate}
		\item Un código $P$ denominado \textbf{espacio de textos planos} y otro código $C$ denominado \textbf{espacio de textos cifrados}.
		\item Un conjunto $K$ denominado \textbf{espacio de claves}.
		\item Un conjunto $\{E_e : P \rightarrow C \mid e \in K\}$ cuyos elementos se denominan \textbf{transformaciones de cifrado} y un conjunto $\{D_d : C \rightarrow P \mid d \in K\}$ cuyos elementos se denominan \textbf{transformaciones de descifrado} tales que, dados un par $(e, d) \in K^2$ con los cuales para todo $p \in P$ se cumple $D_d(E_e(p)) = p$, es computacionalmente difícil obtener $d$ a partir de $e$.
	\end{enumerate}
\end{definition}

\subsubsection{Esquema RSA}

Uno de los esquemas de cifrado asimétrico más frecuentemente utilizados en la actualidad se conoce como RSA. Consiste en tres algoritmos: uno para generar las claves, otro para el cifrado de mensajes, y otro para el descifrado.

\begin{algorithm}
	\caption{Generación de claves}\label{alg:1}
	\KwData{Dos primos $p$ y $q$ cualesquiera. \Comment{Estos dos primos son la \textit{información adicional} que facilitará obtener la preimagen de una función irreversible con trampa.}}
	\KwResult{El par de clave pública y clave privada $((e, n), d)$.}
	$n \gets pq$ \Comment*[r]{Este valor se conoce como \textit{módulo}.}
	$\phi = (p - 1)(q - 1)$\;
	Seleccionar un valor $e$ tal que $1 < e < \phi$ y el menor común múltiplo entre $e$ y $\phi$ sea $1$\;
	Utilizar el algoritmo euclideano extendido para calcular un valor $d$ tal que $ed \equiv 1 \mod \phi$\;
\end{algorithm}

\begin{algorithm}
	\caption{Cifrado}\label{alg:2}
	\KwData{Un mensaje en texto plano $p$ y los valores $n$ y $e$ obtenidos en el Algoritmo 1.}
	\KwResult{El mensaje en texto cifrado $c$.}
	Se computa $m \in [0, p) \cap \integers$ mediante una biyección $f$ tal que $m = f(p)$\;
	Se calcula $c = m^e \mod n$\;
\end{algorithm}

\begin{algorithm}
	\caption{Descifrado}\label{alg:3}
	\KwData{Un mensaje en texto cifrado $c$ y los valores $n$ y $d$ obtenidos en el Algoritmo 1.}
	\KwResult{El mensaje en texto plano $p$.}
	Se calcula $m = c^d \mod n$\;
	Se computa $p = f^{-1}(m)$, siendo $f$ la misma biyección utilizada en el Algoritmo 2.
\end{algorithm}

\newpage

\begin{theorem}
	Sea $c$ un texto cifrado a partir del mensaje $m$. Entonces, el valor obtenido al descifrar $c$ es el mismo valor $m$ de origen.
\end{theorem}

\begin{proof}
	???
\end{proof}

Debido a que el texto cifrado es un valor módulo $n$, la longitud del contenido a cifrar es una limitación de RSA. Este es uno de los motivos por los cuales, en la práctica, el algoritmo no se utiliza para encriptar el mensaje en sí, sino para encriptar las claves del esquema de cifrado simétrico que realmente se utilizan para encriptar el mensaje.

\begin{example}
	Ejemplo con valores numéricos.
\end{example}

Explicar la matemática que hay detrás (DLP), explicar RSA, ElGamal.

\subsubsection{Vulnerabilidades}

Explicar ¿cómo? ¿por qué? funciona el algoritmo de Shor.

\subsection{Fundamentos de la teoría de códigos}

\subsubsection{Códigos de bloque}

\begin{definition}
	Un \textbf{alfabeto} es un conjunto finito no vacío cuyos elementos son denominados \textbf{letras} o \textbf{símbolos}.
\end{definition}

\begin{definition}
	Una \textbf{palabra} sobre un alfabeto $A$ es una $n$-tupla formada por símbolos de $A$. A $n$ se la denomina \textbf{longitud} de la palabra. Si $\textbf{x}$ es una palabra, se denota por $\textbf{x}_i$ la letra de $\textbf{x}$ en la posición $i$.
\end{definition}

\begin{definition}
	Un \textbf{código} es un conjunto de palabras sobre un mismo alfabeto. Un \textbf{código de bloque} es un código cuyas palabras tienen todas la misma longitud.
\end{definition}

\begin{example}
	Sea $A = \mathbb{F}_2$ el alfabeto binario. El conjunto de todas las palabras binarias de longitud $5$ es el código de bloque $C = \mathbb{F}_2^5$. La palabra $\textbf{x} = (0, 1, 1, 0, 1)$ cumple $\textbf{x} \in C$.
\end{example}

\begin{example} \textbf{(Código ISBN).}
	El código ISBN es un identificador único para libros, donde los nueve primeros dígitos de una palabra tienen valor entre $0$ y $9$ y contienen información sobre el código del país, la lengua de origen, el editor o el número del artículo; y el décimo es un \textbf{dígito de control} que verifica $\sum_{i=1}^{10} ia_i \equiv 0 \mod 11$. En términos matemáticos, $C_{\textrm{ISBN}}\subset\mathbb{F}_{11}^{10}$ es un código de bloque de longitud $10$ tal que
	\[C_{\textrm{ISBN}} = \{\textbf{x}\in\mathbb{F}_{11}^{10} \mid (a_i = 10 \implies i = 10) \wedge \sum_{i=1}^{10}ia_i \equiv 0 \mod 11\}.\]
	\begin{remark}
		Cuando $a_{10} = 10$, suele representarse con el símbolo $\textrm{X}$, pero a efectos de cálculo toma valor $10$. Por ejemplo, $\textrm{417339361X} \in C_{\textrm{ISBN}}$.
	\end{remark}
\end{example}

\begin{definition}
	Sean $\textbf{x}$ e $\textbf{y}$ dos palabras de misma longitud. La \textbf{distancia de Hamming} entre ambas es igual al número de letras en que difieren. Dicho de otra forma, si $\textbf{x} = (x_1, \hdots, x_n)$ y $\textbf{y} = (y_1, \hdots, y_n)$, entonces
	\[d(\textbf{x}, \textbf{y}) = |\{i\in[1, n]\cap\naturals \mid x_i \neq y_i\}|,\]
	donde $d(\textbf{x}, \textbf{y})$ es la distancia de Hamming entre $\textbf{x}$ e $\textbf{y}$.
\end{definition}

\begin{theorem}
	La distancia de Hamming entre  $\textbf{x}$ e $\textbf{y}$ tales que $\textbf{x}, \textbf{y} \in A^n$ es una distancia definida en $A^n$.
\end{theorem}

\begin{proof}
	Se demuestran a continuación las cuatro propiedades de las distancias:
	\begin{enumerate}
		\item \textit{La distancia de una palabra a sí misma es $0$}: en efecto,
		\[d(\textbf{x}, \textbf{x}) = |\{i\in[1,n]\cap\naturals \mid x_i \neq x_i\}| = |\{\}| = 0.\]
		\item \textit{La distancia entre dos palabras distintas es positiva}: si $\textbf{x}$ e $\textbf{y}$ son palabras distintas, significa que las dos tuplas deben diferir en al menos una de las posiciones; es decir, debe existir al menos un valor de $i$ tal que $x_i \neq y_i$, por lo que la cardinalidad de $\{i \in [1, n]\cap\naturals \mid x_i \neq y_i\}$ será como mínimo de $1$.
		\item \textit{La distancia de $\textbf{x}$ a $\textbf{y}$ es la misma que de $\textbf{y}$ a $\textbf{x}$}: se verifica
		\[d(\textbf{x}, \textbf{y}) = |\{i\in[1,n]\cap\naturals \mid x_i \neq y_i\}| = |\{i\in[1,n]\cap\naturals \mid y_i \neq x_i\}| = d(\textbf{y}, \textbf{x}).\]
		\item \textit{Se cumple la desigualdad triangular $d(\textbf{x}, \textbf{z}) \leq d(\textbf{x}, \textbf{y}) + d(\textbf{y}, \textbf{z})$}:
		Se define
		\[\Delta(\textbf{x}, \textbf{y}) = \{i\in[1,n]\cap\naturals \mid x_i \neq y_i\}.\]
		
		Supóngase que existen $\textbf{x}, \textbf{y}, \textbf{z} \in A^n$ tales que $d(\textbf{x}, \textbf{z}) > d(\textbf{x}, \textbf{y}) + d(\textbf{y}, \textbf{z})$. Entonces, debe existir $i \in [1, n]\cap\naturals$ tal que $i \in \Delta(\textbf{x}, \textbf{z})$, $i \notin \Delta(\textbf{x}, \textbf{y})$ e $i \notin \Delta(\textbf{y}, \textbf{z})$. Eso implica $x_i \neq z_i$ a la vez que $x_i = y_i = z_i$, lo cual es una contradicción.
	\end{enumerate}
\end{proof}

\begin{definition}
	Sea $\textbf{x}$ una palabra de longitud $n$. Se llama \textbf{peso de $\textbf{x}$}, y se denomina por $w(\textbf{x})$, al número de elementos de $\textbf{x}$ distintos de $0$.
\end{definition}

\begin{definition}
	El \textbf{peso mínimo} de un código $C$, denominado por $w(C)$, es el menor de los pesos no nulos entre las palabras diferentes de $C$. Es decir,
	\[w(C) = \min\{w(\textbf{x}) \mid \textbf{x}\in (C - 0)\}.\]
\end{definition}

\begin{definition}
	La \textbf{distancia mínima} de un código $C \subseteq A^n$ es la menor de las distancias entre palabras diferentes de $C$ si $|C| > 1$, y $n+1$ en caso contrario; es decir, para códigos formados por más de una palabra,
	\[d(C) = \min\{d(\textbf{x}, \textbf{y}) \mid \textbf{x}, \textbf{y} \in C, \textbf{x}\neq\textbf{y}\}.\]
	
	\begin{remark}
		Se denotará la distancia mínima simplemente por $d$ cuando por contexto esté claro a qué código corresponde. Así, por norma general, $d = d(C)$.
	\end{remark}
\end{definition}

\begin{definition}
	Sea $\textbf{x} \in A^n$. La \textbf{bola cerrada de centro $\textbf{x}$ y radio $r$} se denota por $B_r(\textbf{x})$ y se define como
	\[B_r(\textbf{x}) = \{\textbf{y} \in A^n \mid d(\textbf{x}, \textbf{y}) \leq r\}.\]
	Similarmente, se denota por $S_r(\textbf{x})$ a la \textbf{esfera de centro $\textbf{x}$ y radio $r$}, y queda definida como
	\[S_r(\textbf{x}) = \{\textbf{y} \in A^n \mid d(\textbf{x}, \textbf{y}) = r\}.\]
	\begin{remark}
		Ya que la distancia de Hamming devuelve únicamente números naturales, es inmediata la siguiente afirmación:
		\[B_r(\textbf{x}) = \bigcup_{i=0}^r S_i(\textbf{x}).\]
	\end{remark}
\end{definition}

\begin{theorem}
	Sea $A$ un alfabeto con $q$ elementos y $\textbf{x} \in A^n$. Entonces,
	\[|B_r(\textbf{x})| = \sum_{i=0}^r\binom{n}{i}(q - 1)^i.\]
\end{theorem}

\begin{proof}
	Se define
	\[\Delta(\textbf{x}, \textbf{y}) = \{i\in[1,n]\cap\naturals \mid x_i \neq y_i\}.\]
	Sea $\textbf{y} \in S_i(\textbf{x})$. Por definición, $i = |\Delta(\textbf{x}, \textbf{y})|$.
	
	Ya que las palabras tienen longitud $n$, y se realizan $i$ cambios para llegar de $\textbf{x}$ a $\textbf{y}$, existen exactamente $\binom{n}{i}$ conjuntos $\Delta(\textbf{x}, \textbf{y})$ que verifican las ecuaciones de arriba.
	
	Además, para un $\Delta(\textbf{x}, \textbf{y})$ arbitrario, existen $(q - 1)^i$ posibles palabras $\textbf{y} \in S_i(\textbf{x})$ que verifican las ecuaciones.
	
	Así, el número de $\textbf{y} \in S_i(\textbf{x})$ que podrían verificar las ecuaciones es $\binom{n}{i}(q - 1)^i$; es decir,
	\[|S_i(\textbf{x})| = \binom{n}{i}(q - 1)^i.\]
	
	Finalmente, por la \textit{Nota 3}, se concluye que
	\[|B_r(\textbf{x})| = \bigg|\bigcup_{i=0}^r S_i(\textbf{x})\bigg| = \sum_{i=0}^r |S_i(\textbf{x})| = \sum_{i=0}^r\binom{n}{i}(q - 1)^i.\]
\end{proof}

\subsubsection{Códigos lineales}

\begin{definition}
	Un \textbf{código lineal} es un código de bloque con estructura de subespacio vectorial; es decir, para un código lineal $C$ sobre $\mathbb{F}_q^n$ se cumplen las siguientes propiedades:
	\begin{enumerate}
		\item El código $C$ no es vacío.
		\item Para todos $\textbf{x}, \textbf{y} \in C$ se verifica $\textbf{x} + \textbf{y} \in C$.
		\item Para todos $\lambda \in \mathbb{F}_q, \textbf{x} \in C$ se verifica $\lambda\textbf{x} \in C$.
	\end{enumerate}

	A un código lineal de longitud $n$ y dimensión $k$ se lo denomina \textbf{$(n, k)$-código}.
\end{definition}

\begin{theorem}
	Sea $C$ un código lineal. Se verifica $d(C) = w(C)$.
\end{theorem}

\begin{proof}
	Sean $\textbf{x}, \textbf{y} \in C$. Ya que $C$ es un código lineal, se cumple $\textbf{x} - \textbf{y} \in C$. Obsérvese que $\textbf{x}_i = \textbf{y}_i$ implica $x_i - y_i = 0$, luego $w(\textbf{x} - \textbf{y}) = d(\textbf{x}, \textbf{y})$. Así, la distancia entre cualesquiera dos palabras es siempre igual al peso de otra, y el peso de cualquier palabra es siempre igual a la distancia entre dos otras palabras, por lo que el valor menor debe ser igual en ambos casos.
\end{proof}

\begin{definition}
	Una \textbf{base} de un $(n, k)$-código $C$ es una $k$-tupla de palabras de $C$ tal que cada palabra de $C$ es combinación lineal de los elementos de la base.
\end{definition}

\begin{definition}
	Una \textbf{matriz generadora} de un $(n, k)$-código es una matriz cuyas filas forman una base de $C$.
\end{definition}

\subsubsection{Códigos cíclicos}

\begin{definition}
	El \textbf{desplazamiento cíclico} de una palabra $\textbf{x} \in \mathbb{F}_q^n$, denotado por $\sigma(\textbf{x})$, es una transformación tal que
	\[\sigma(\textbf{x}) = (\textbf{x}_{n-1}, \textbf{x}_0, \textbf{x}_1, \dots, \textbf{x}_{n-2}).\]
\end{definition}

\begin{definition}
	Un \textbf{código cíclico} es un código que contiene el desplazamiento cíclico de cada una de sus palabras.
\end{definition}

\subsubsection{Códigos polinómicos}

\subsubsection{Códigos Goppa}

\section{Trabajos relacionados}

(Sobre los algoritmos BIKE, Classic McEliece y HQC).
