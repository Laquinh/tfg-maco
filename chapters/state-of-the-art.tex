\chapter{Estado de la cuestión}  
\addcontentsline{toc}{chapter}{\numberline{}Estado de la cuestión}

\section{Marco teórico del trabajo}

\subsection{Esquemas de cifrado asimétrico}

\subsubsection{Fundamentos}

\begin{definition}
	Una función $f : X \rightarrow Y$ es \textbf{irreversible} si para un $x \in X$ es computacionalmente fácil\footnote{Si bien este concepto puede ser definido de una manera más rigurosa, ya que esto queda fuera del ámbito de este trabajo, entiéndase por \textit{computacionalmente difícil} que el número de operaciones que hacen falta para obtener $d$ a partir de $e$ es tan grande que resulta inviable llevarlo a la práctica.} calcular $f(x)$ pero para un $y \in Im(f)$ es computacionalmente difícil obtener un valor $x$ tal que $f(x) = y$, y por \textit{computacionalmente fácil} que este caso no ocurra.
\end{definition}

\begin{definition}
	Una función irreversible $f : X \rightarrow Y$ es \textbf{irreversible con trampa} si existe información adicional con la cual sea computacionalmente fácil para un $y \in Im(f)$ obtener un valor $x$ tal que $f(x) = y$.
\end{definition}

\begin{definition}
	Un \textbf{esquema de cifrado asimétrico} está dado por
	\begin{enumerate}
		\item Un código $P$ denominado \textbf{espacio de textos planos} y otro código $C$ denominado \textbf{espacio de textos cifrados}.
		\item Un conjunto $K$ denominado \textbf{espacio de claves}.
		\item Un conjunto $\{E_e : P \rightarrow C \mid e \in K\}$ cuyos elementos se denominan \textbf{transformaciones de cifrado} y un conjunto $\{D_d : C \rightarrow P \mid d \in K\}$ cuyos elementos se denominan \textbf{transformaciones de descifrado} tales que, dados un par $(e, d) \in K$ con los cuales para todo $p \in P$ se cumple $D_d(E_e(p)) = p$, es computacionalmente difícil obtener $d$ a partir de $e$.
	\end{enumerate}
\end{definition}

\subsubsection{Esquema RSA}

Uno de los esquemas de cifrado asimétrico más frecuentemente utilizados en la actualidad se conoce como RSA. Consiste en tres algoritmos: uno para generar las claves, otro para el cifrado de mensajes, y otro para el descifrado.

\begin{algorithm}
	\caption{Generación de claves}\label{alg:1}
	\KwData{Dos primos $p$ y $q$ cualesquiera. \Comment{Estos dos primos son la \textit{información adicional} que facilitará obtener la preimagen de una función irreversible con trampa.}}
	\KwResult{El par de clave pública y clave privada $((e, n), d)$.}
	$n \gets pq$ \Comment*[r]{Este valor se conoce como \textit{módulo}.}
	$\phi = (p - 1)(q - 1)$\;
	Seleccionar un valor $e$ tal que $1 < e < \phi$ y el menor común múltiplo entre $e$ y $\phi$ sea $1$\;
	Utilizar el algoritmo euclideano extendido para calcular un valor $d$ tal que $eq \equiv 1 \mod \phi$\;
\end{algorithm}

\begin{algorithm}
	\caption{Cifrado}\label{alg:2}
	\KwData{Un mensaje en texto plano $p$ y los valores $n$ y $e$ obtenidos en el Algoritmo 1.}
	\KwResult{El mensaje en texto cifrado $c$.}
	Se computa $m \in [0, p) \cap \integers$ mediante una biyección $f$ tal que $m = f(p)$\;
	Se calcula $c = m^e \mod n$\;
\end{algorithm}

\begin{algorithm}
	\caption{Descifrado}\label{alg:3}
	\KwData{Un mensaje en texto cifrado $c$ y los valores $n$ y $d$ obtenidos en el Algoritmo 1.}
	\KwResult{El mensaje en texto plano $p$.}
	Se calcula $m = c^d \mod n$\;
	Se computa $p = f^{-1}(m)$, siendo $f$ la misma biyección utilizada en el Algoritmo 2.
\end{algorithm}

\begin{theorem}
	Sea $c$ un texto cifrado a partir del mensaje $m$. Entonces, el valor obtenido al descifrar $c$ es el mismo valor $m$ de origen.
\end{theorem}

\begin{proof}
	???
\end{proof}

\begin{example} \textbf{(Problema del logaritmo discreto (DLP)).}
	Sea $g : \mathbb{F}_q^* \rightarrow \mathbb{F}_q^*$ tal que $g(x) = a^x$ y $a$ genere el grupo multiplicativo $\mathbb{F}_q^*$. Se cree que obtener una función inversa de $g$ es un problema computacionalmente difícil.
\end{example}

Explicar la matemática que hay detrás (DLP), explicar RSA, ElGamal.

\subsubsection{Vulnerabilidades}

Explicar ¿cómo? ¿por qué? funciona el algoritmo de Shor.

\subsection{Fundamentos de la teoría de códigos}

\subsubsection{Códigos de bloque}

\begin{definition}
	Un \textbf{alfabeto} es un conjunto finito no vacío cuyos elementos son denominados \textbf{letras} o \textbf{símbolos}.
\end{definition}

\begin{definition}
	Una \textbf{palabra} sobre un alfabeto $A$ es una $n$-tupla formada por símbolos de $A$. A $n$ se la denomina \textbf{longitud} de la palabra.
\end{definition}

\begin{definition}
	Un \textbf{código} es un conjunto de palabras sobre un mismo alfabeto. Un \textbf{código de bloque} es un código cuyas palabras tienen todas la misma longitud.
\end{definition}

\begin{example}
	Sea $A = \mathbb{F}_2$ el alfabeto binario. El conjunto de todas las palabras binarias de longitud $5$ es el código de bloque $C = \mathbb{F}_2^5$. La palabra $\textbf{x} = (0, 1, 1, 0, 1)$ cumple $\textbf{x} \in C$.
\end{example}

\begin{example} \textbf{(Código ISBN).}
	El código ISBN es un identificador único para libros, donde los nueve primeros dígitos de una palabra tienen valor entre $0$ y $9$ y contienen información sobre el código del país, la lengua de origen, el editor o el número del artículo; y el décimo es un \textbf{dígito de control} que verifica $\sum_{i=1}^{10} ia_i \equiv 0 \mod 11$. En términos matemáticos, $C_{\textrm{ISBN}}\subset\mathbb{F}_{11}^{10}$ es un código de bloque de longitud $10$ tal que
	\[C_{\textrm{ISBN}} = \{\textbf{x}\in\mathbb{F}_{11}^{10} \mid (a_i = 10 \implies i = 10) \wedge \sum_{i=1}^{10}ia_i \equiv 0 \mod 11\}.\]
	\begin{remark}
		Cuando $a_{10} = 10$, suele representarse con el símbolo $\textrm{X}$, pero a efectos de cálculo toma valor $10$. Por ejemplo, $\textrm{417339361X} \in C_{\textrm{ISBN}}$.
	\end{remark}
\end{example}

\begin{definition}
	Sean $\textbf{x}$ e $\textbf{y}$ dos palabras de misma longitud. La \textbf{distancia de Hamming} entre ambas es igual al número de letras en que difieren. Dicho de otra forma, si $\textbf{x} = (x_1, \hdots, x_n)$ y $\textbf{y} = (y_1, \hdots, y_n)$, entonces
	\[d(\textbf{x}, \textbf{y}) = |\{i\in[1, n]\cap\naturals \mid x_i \neq y_i\}|,\]
	donde $d(\textbf{x}, \textbf{y})$ es la distancia de Hamming entre $\textbf{x}$ e $\textbf{y}$.
\end{definition}

\begin{theorem}
	La distancia de Hamming entre  $\textbf{x}$ e $\textbf{y}$ tales que $\textbf{x}, \textbf{y} \in A^n$ es una distancia definida en $A^n$.
\end{theorem}

\begin{proof}
	Se demuestran a continuación las cuatro propiedades de las distancias:
	\begin{enumerate}
		\item \textit{La distancia de una palabra a sí misma es $0$}: en efecto,
		\[d(\textbf{x}, \textbf{x}) = |\{i\in[1,n]\cap\naturals \mid x_i \neq x_i\}| = |\{\}| = 0.\]
		\item \textit{La distancia entre dos palabras distintas es positiva}: si $\textbf{x}$ e $\textbf{y}$ son palabras distintas, significa que las dos tuplas deben diferir en al menos una de las posiciones; es decir, debe existir al menos un valor de $i$ tal que $x_i \neq y_i$, por lo que la cardinalidad de $\{i \in [1, n]\cap\naturals \mid x_i \neq y_i\}$ será como mínimo de $1$.
		\item \textit{La distancia de $\textbf{x}$ a $\textbf{y}$ es la misma que de $\textbf{y}$ a $\textbf{x}$}: se verifica
		\[d(\textbf{x}, \textbf{y}) = |\{i\in[1,n]\cap\naturals \mid x_i \neq y_i\}| = |\{i\in[1,n]\cap\naturals \mid y_i \neq x_i\}| = d(\textbf{y}, \textbf{x}).\]
		\item \textit{Se cumple la desigualdad triangular $d(\textbf{x}, \textbf{z}) \leq d(\textbf{x}, \textbf{y}) + d(\textbf{y}, \textbf{z})$}:
		Se define
		\[\Delta(\textbf{x}, \textbf{y}) = \{i\in[1,n]\cap\naturals \mid x_i \neq y_i\}.\]
		
		Supóngase que existen $\textbf{x}, \textbf{y}, \textbf{z} \in A^n$ tales que $d(\textbf{x}, \textbf{z}) > d(\textbf{x}, \textbf{y}) + d(\textbf{y}, \textbf{z})$. Entonces, debe existir $i \in [1, n]\cap\naturals$ tal que $i \in \Delta(\textbf{x}, \textbf{z})$, $i \notin \Delta(\textbf{x}, \textbf{y})$ e $i \notin \Delta(\textbf{y}, \textbf{z})$. Eso implica $x_i \neq z_i$ a la vez que $x_i = y_i = z_i$, lo cual es una contradicción.
	\end{enumerate}
\end{proof}

\begin{definition}
	La \textbf{distancia mínima} de un código $C \subseteq A^n$ es la menor de las distancias entre palabras diferentes de $C$ si $|C| > 1$, y $n+1$ en caso contrario; es decir, para códigos formados por más de una palabra,
	\[d(C) = \min\{d(\textbf{x}, \textbf{y}) \mid \textbf{x}, \textbf{y} \in C, \textbf{x}\neq\textbf{y}\}.\]
	
	\begin{remark}
		Se denotará la distancia mínima simplemente por $d$ cuando por contexto esté claro a qué código corresponde. Así, por norma general, $d = d(C)$.
	\end{remark}
\end{definition}

\begin{definition}
	Sea $\textbf{x} \in A^n$. La \textbf{bola cerrada de centro $\textbf{x}$ y radio $r$} se denota por $B_r(\textbf{x})$ y se define como
	\[B_r(\textbf{x}) = \{\textbf{y} \in A^n \mid d(\textbf{x}, \textbf{y}) \leq r\}.\]
	Similarmente, se denota por $S_r(\textbf{x})$ a la \textbf{esfera de centro $\textbf{x}$ y radio $r$}, y queda definida como
	\[S_r(\textbf{x}) = \{\textbf{y} \in A^n \mid d(\textbf{x}, \textbf{y}) = r\}.\]
	\begin{remark}
		Ya que la distancia de Hamming devuelve únicamente números naturales, es inmediata la siguiente afirmación:
		\[B_r(\textbf{x}) = \bigcup_{i=0}^r S_i(\textbf{x}).\]
	\end{remark}
\end{definition}

\begin{theorem}
	Sea $A$ un alfabeto con $q$ elementos y $\textbf{x} \in A^n$. Entonces,
	\[|B_r(\textbf{x})| = \sum_{i=0}^r\binom{n}{i}(q - 1)^i.\]
\end{theorem}

\begin{proof}
	Se define
	\[\Delta(\textbf{x}, \textbf{y}) = \{i\in[1,n]\cap\naturals \mid x_i \neq y_i\}.\]
	Sea $\textbf{y} \in S_i(\textbf{x})$. Por definición, $i = |\Delta(\textbf{x}, \textbf{y})|$.
	
	Ya que las palabras tienen longitud $n$, y se realizan $i$ cambios para llegar de $\textbf{x}$ a $\textbf{y}$, existen exactamente $\binom{n}{i}$ conjuntos $\Delta(\textbf{x}, \textbf{y})$ que verifican las ecuaciones de arriba.
	
	Además, para un $\Delta(\textbf{x}, \textbf{y})$ arbitrario, existen $(q - 1)^i$ posibles palabras $\textbf{y} \in S_i(\textbf{x})$ que verifican las ecuaciones.
	
	Así, el número de $\textbf{y} \in S_i(\textbf{x})$ que podrían verificar las ecuaciones es $\binom{n}{i}(q - 1)^i$; es decir,
	\[|S_i(\textbf{x})| = \binom{n}{i}(q - 1)^i.\]
	
	Finalmente, por la \textit{Nota 3}, se concluye que
	\[|B_r(\textbf{x})| = \bigg|\bigcup_{i=0}^r S_i(\textbf{x})\bigg| = \sum_{i=0}^r |S_i(\textbf{x})| = \sum_{i=0}^r\binom{n}{i}(q - 1)^i.\]
\end{proof}

\subsubsection{Códigos lineales}

\subsubsection{Códigos cíclicos}

\subsubsection{Códigos polinómicos}

\subsubsection{Códigos Goppa}

\section{Trabajos relacionados}

(Sobre los algoritmos BIKE, Classic McEliece y HQC).
